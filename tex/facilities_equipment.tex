Since this robot is required to work underwater, we will need a pool of some sort to test it out. The pool must be large enough and deep enough to put stuff in and for our robot to move around in. Its possible we can just buy a pool or contact someone who has a pool that we can use. Another thing we need is a 3D printer to create custom parts. We will be using the application Tinkercad for 3D modeling. The parts for the robot will be purchased online. We have put together a list of equipment that we will start out with and add more things to the list as project testing continues. We will need 6 pieces of carbon fiber tubing, which will be the basis for the frame. Another thing we need is a pelican case, a watertight container to hold our electronics. Lastly, we will need an Arduino Uno R3, a microcontroller we might change in the future for more outputs. One thing we are on the fence on is if we should get Ballast tanks or thrusters. We're leaning more on thrusters as they are easier to use and have more accessible.