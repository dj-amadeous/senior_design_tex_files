Currently as this is consisting of the first round of parts, the group was conservative with their purchasing with the initial budget of \$800.00. The Arduino was needed as a basic microcontroller to handle future thruster implementation. Dr. McMurrough has thrusters that he will lend for the project, so the microcontroller will be able to interface directly into those. Expansions to the microcontroller could be needed like an I2C bus for more inputs, but that is undetermined at this time. Next the chosen frame was to make it out of glossy carbon fiber tubing. Six total pieces were purchased to give the group the initial flexibility to create a basic frame to mount components to. The pelican watertight case was another recommendation by Dr. McMurrough to hold the electronics and keep them safe. Possible weights or concerns of density will be addressed when the parts arrive.



\subsection{Preliminary Budget}
\begin{table}[H]
	\resizebox{\textwidth}{!}{
		\begin{tabular}{|l|l|l|}
			\hline
			\textbf{Purchased Item} & \textbf{Quantity} & \textbf{Cost} \\ \hline
			Arduino Uno R3  & 1 & \$13.99 \\ \hline
			Carbon Fiber Tube 16mm x 14mm x 500mm (2 Pieces)  & 3 & \$56.85  \\ \hline
			Pelican 1200 Case With Foam & 1 & \$52.72 \\ \hline
			Total Cost: & - & \$123.56 \\ \hline
	\end{tabular}}
	\caption{Breakdown of Current Costs} 
\end{table}



\subsection{Current \& Pending Support}
Senior Design Standard Budget \$800.00 - This is given to all senior design teams to use for the progression of their project. Since this project is hardware based, it is possible that this budget could increase if the need arises, but for now the budget is firm \$800.00.